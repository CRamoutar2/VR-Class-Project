

\documentclass{vgtc}                          % final (conference style)
%\documentclass[review]{vgtc}                 % review
%\documentclass[widereview]{vgtc}             % wide-spaced review
%\documentclass[preprint]{vgtc}               % preprint
%\documentclass[electronic]{vgtc}             % electronic version

%% Uncomment one of the lines above depending on where your paper is
%% in the conference process. ``review'' and ``widereview'' are for review
%% submission, ``preprint'' is for pre-publication, and the final version
%% doesn't use a specific qualifier. Further, ``electronic'' includes
%% hyperreferences for more convenient online viewing.

%% Please use one of the ``review'' options in combination with the
%% assigned online id (see below) ONLY if your paper uses a double blind
%% review process. Some conferences, like IEEE Vis and InfoVis, have NOT
%% in the past.

%% Figures should be in CMYK or Grey scale format, otherwise, colour 
%% shifting may occur during the printing process.

%% These few lines make a distinction between latex and pdflatex calls and they
%% bring in essential packages for graphics and font handling.
%% Note that due to the \DeclareGraphicsExtensions{} call it is no longer necessary
%% to provide the the path and extension of a graphics file:
%% \includegraphics{diamondrule} is completely sufficient.
%%
\ifpdf%                                % if we use pdflatex
  \pdfoutput=1\relax                   % create PDFs from pdfLaTeX
  \pdfcompresslevel=9                  % PDF Compression
  \pdfoptionpdfminorversion=7          % create PDF 1.7
  \ExecuteOptions{pdftex}
  \usepackage{graphicx}                % allow us to embed graphics files
  \DeclareGraphicsExtensions{.pdf,.png,.jpg,.jpeg} % for pdflatex we expect .pdf, .png, or .jpg files
\else%                                 % else we use pure latex
  \ExecuteOptions{dvips}
  \usepackage{graphicx}                % allow us to embed graphics files
  \DeclareGraphicsExtensions{.eps}     % for pure latex we expect eps files
\fi%

%% it is recomended to use ``\autoref{sec:bla}'' instead of ``Fig.~\ref{sec:bla}''
\graphicspath{{figures/}{pictures/}{images/}{./}} % where to search for the images

\usepackage{microtype}                 % use micro-typography (slightly more compact, better to read)
\PassOptionsToPackage{warn}{textcomp}  % to address font issues with \textrightarrow
\usepackage{textcomp}                  % use better special symbols
\usepackage{mathptmx}                  % use matching math font
\usepackage{times}                     % we use Times as the main font
\renewcommand*\ttdefault{txtt}         % a nicer typewriter font
\usepackage{cite}                      % needed to automatically sort the references
%% We encourage the use of mathptmx for consistent usage of times font
%% throughout the proceedings. However, if you encounter conflicts
%% with other math-related packages, you may want to disable it.


%% If you are submitting a paper to a conference for review with a double
%% blind reviewing process, please replace the value ``0'' below with your
%% OnlineID. Otherwise, you may safely leave it at ``0''.
\onlineid{0}

%% declare the category of your paper, only shown in review mode
\vgtccategory{Research}

%% allow for this line if you want the electronic option to work properly
\vgtcinsertpkg

%% In preprint mode you may define your own headline.
%\preprinttext{To appear in an IEEE VGTC sponsored conference.}

%% Paper title.

\title{Group 5 Project Proposal}

%% This is how authors are specified in the conference style

%% Author and Affiliation (single author).
%%\author{Roy G. Biv\thanks{e-mail: roy.g.biv@aol.com}}
%%\affiliation{\scriptsize Allied Widgets Research}

%% Author and Affiliation (multiple authors with single affiliations).
%%\author{Roy G. Biv\thanks{e-mail: roy.g.biv@aol.com} %
%%\and Ed Grimley\thanks{e-mail:ed.grimley@aol.com} %
%%\and Martha Stewart\thanks{e-mail:martha.stewart@marthastewart.com}}
%%\affiliation{\scriptsize Martha Stewart Enterprises \\ Microsoft Research}

%% Author and Affiliation (multiple authors with multiple affiliations)
\makeatletter
\newcommand{\linebreakand}{%
  \end{@IEEEauthorhalign}
  \hfill\mbox{}\par
  \mbox{}\hfill\begin{@IEEEauthorhalign}
}
\makeatother

\author{\IEEEauthorblockN{1\textsuperscript{st} Chethram Ramoutar}
  \IEEEauthorblockA{\textit{Hunter College} \\
  chethram.ramoutar79@myhunter.cuny.edu}
  \and
  \IEEEauthorblockN{2\textsuperscript{nd} Mark Blinder}
  \IEEEauthorblockA{\textit{Hunter College} \\
  mark.blinder54@myhunter.cuny.edu}\\
  \and
  \IEEEauthorblockN{3\textsuperscript{rd} Omar Hegazy}
  \IEEEauthorblockA{\textit{Hunter College} \\
  omar.hegazy00@myhunter.cuny.edu}
  \and
  \IEEEauthorblockN{4\textsuperscript{th} Melanie Vasquez}
  \IEEEauthorblockA{\textit{Hunter College} \\
    melanie.vasquez61@myhunter.cuny.edu}
}

%% A teaser figure can be included as follows, but is not recommended since
%% the space is now taken up by a full width abstract.
%\teaser{
%  \includegraphics[width=1.5in]{sample.eps}
%  \caption{Lookit! Lookit!}
%}

%% Abstract section.
\abstract{
With the growth of VR devices being more accessible to the masses, it is understood that environment,
interactivity, and the ability to immerse oneself in a virtual world is a key factor in the success of any
VR experience. However, achieving this experience is often difficult for many games due to lack of attention to interactivity
and immersion. In order to combat this, our group seeks to develop a VR game that will use hand input to immerse a user
in a quest-like game. To achieve immersion and interactivity we will use different tools and methodologies to help the player feel
like they are apart of the experience.%
} % end of abstract

%% ACM Computing Classification System (CCS). 
%% See <http://www.acm.org/about/class> for details.
%% We recommend the 2012 system <http://www.acm.org/about/class/class/2012>
%% For the 2012 system use the ``\CCScatTwelve'' which command takes four arguments.
%% The 1998 system <http://www.acm.org/about/class/class/2012> is still possible
%% For the 1998 system use the ``\CCScat'' which command takes four arguments.
%% In both cases the last two arguments (1998) or last three (2012) can be empty.


%% Copyright space is enabled by default as required by guidelines.
%% It is disabled by the 'review' option or via the following command:
% \nocopyrightspace

%%%%%%%%%%%%%%%%%%%%%%%%%%%%%%%%%%%%%%%%%%%%%%%%%%%%%%%%%%%%%%%%
%%%%%%%%%%%%%%%%%%%%%% START OF THE PAPER %%%%%%%%%%%%%%%%%%%%%%
%%%%%%%%%%%%%%%%%%%%%%%%%%%%%%%%%%%%%%%%%%%%%%%%%%%%%%%%%%%%%%%%%

\begin{document}

%% The ``\maketitle'' command must be the first command after the
%% ``\begin{document}'' command. It prepares and prints the title block.

%% the only exception to this rule is the \firstsection command
\firstsection{Introduction}

\maketitle

%% \section{Introduction} %for journal use above \firstsection{..} instead
The game centers around a lone space traveler who is on a quest to collect different items to build an essential part. As they travel through
different dimensions, the user will experience different terrains, forms of life, and interesting settings. This game will be broken up into four key
areas of travel being: A desert, a forest, a village, and a farm. Each area's focus will allow us to give the user multiple cohesive experiences without
the risk of throwing them off from the main story. This will also allow the developers to test different aspects of each environment to present the idea that
a user truly traveled a different dimension.

Along with the environment, we seek to have non-playable character or NPC interaction that will make each space more lively and natural to the user. However, the main
functionality and interactivity will be with the quests assigned in each area. The user will be able to collect items that relate to their respective quest and explore a majority
of their environment to seek these out. Although it is simple, it can allow the user to immerse themselves without streinuous tasks breaking their immersion. The quests will be entirely focused on
hand input allowing a user to pick up, examine, and collect objects with ease. The user will also be able to look around and navigate the each environment as naturally as possible.

\section{Related Work}
Games that inspired the creation of this project are "The Witness" and "The Forest". "The Witness" uses environment and puzzle interaction in a seemless way. Integration of this in VR has translated well
since the game was first published. By looking at how "The Witness" makes the environment the main aspect, it is something we want to take away from. Despite most if not all of the environment being in low poly 3D graphics, the 
use of ray tracing, saturated colors, and a lack of clutter makes the environment feel alive and natural. The game also uses a first person perspective which is something we want to take away from. The first person perspective allows for a more 
personal and intimate experience with the puzzles that is not often seen in VR games. There is more naturality with this view then with third person perspective.

"The Forest" also uses the previous aspects but in a more realistic and intense setting. The main aspects to take away from is the environmental interaction. Although "The Witness" achieves the appropriate aesthetic while integrating puzzles, "The Forest"
is more realistic with its gameplay. Having the ability to interact with tools such as axes, sticks, and even a bow and arrow brings a lot into the immersion of the game. Movement becomes key in the gameplay which is translated through the well thought out head and hand tracking
that makes every swing or flick feel meaningful in the environment the player is in. Since it is also a survival game, first person perspective brings out the sense of urgency
with every interaction. The game also contains well thought out NPC interactions that are extremely heart racing in a VR environment. By seeing how "The Forest" uses interactivity in its gameplay we want to try and achieve that within our game through our collection
quests. However, we want to keep the calmness and aesthic found in "The Witness" to make the game more enjoyable.

\section{Methodology}
In order to complete the project, it will be done on Unity 2021.3.7f1. All testing will occur with the Oculus Quest 1 and the Oculus Quest 2. Each scene will utilize assets from the Unity Asset Store and any other
source containing appropriate assets. The scenes will be connected through a main hub area which will contain teleportation pads to each area. The user will be able to teleport by standing on a pad and triggering
a script or pressing a button. These portals will also be placed in each area
in order to travel back to allow the user to have a choice on where to go.This will allow us to merge the scenes in a manner that fits with the story but also reduces any streinuous load times that may
occur.

Outside of the hub being worked on, each group member will handle a scene and quest that corresponds with it. The scenes will be split as so:
\begin{itemize}
  \item Desert: Melanie Vasquez
  \item Forest: Mark Blinder
  \item Village: Omar Hegazy
  \item Farm: Chethram Ramoutar
\end{itemize}

Interactivity will be integrated into each scene through the use of the XR rig and XR Interaction Toolkit. This rig will allow for VR testing due to the headset detection built into it and controller support allowing us to add eye and hand detection.
Along with that, the use of the XR Device Simulator will help with non-vr testing to take place throughout the debugging process since we can tweak any features without the full need of a VR headser. The XR Interaction Toolkit will allow for the user to
interact with objects in the scene by using their hands or eyes and will help us with the main portion of the interactivity. Since we are using hands as the main input device, having this allows us to test and enhance interaction experience with either haptic feedback
or visual feedback. This can help build upon each scene once we get into NPC interactions as well and ways the user can have the illusion of communication within each area.

Outside of the interactivity, for immersion we will use assets and different light and rendering techniques to allow a user to be in a space that is not dull. Animations will also be used because it can give the illusion of movement which can add life to each scene.
These animations may be applied to the collected objects, the NPCs, or even the environment itself but overall the main focus is NPC animations. This can be done through Mixamo and scripts that trigger certain movements within a player model.



%% if specified like this the section will be committed in review mode
\acknowledgments{
  The authors wish to thank A, B, and C. This work was supported in part by
  a grant from XYZ.}

%\bibliographystyle{abbrv}
\bibliographystyle{abbrv-doi}
%\bibliographystyle{abbrv-doi-narrow}
%\bibliographystyle{abbrv-doi-hyperref}
%\bibliographystyle{abbrv-doi-hyperref-narrow}

\bibliography{template}
\end{document}
