\documentclass[conference]{IEEEtran}
\IEEEoverridecommandlockouts
% The preceding line is only needed to identify funding in the first footnote. If that is unneeded, please comment it out.
\usepackage{cite}
\usepackage{amsmath,amssymb,amsfonts}
\usepackage{algorithmic}
\usepackage{graphicx}
\usepackage{textcomp}
\usepackage{xcolor}
\def\BibTeX{{\rm B\kern-.05em{\sc i\kern-.025em b}\kern-.08em
    T\kern-.1667em\lower.7ex\hbox{E}\kern-.125emX}}
\begin{document}

\title{Group 5 Project Proposal\\
}

% Helps create proper centering for this new linebreak
% Received from stackexchange
\makeatletter
\newcommand{\linebreakand}{%
  \end{@IEEEauthorhalign}
  \hfill\mbox{}\par
  \mbox{}\hfill\begin{@IEEEauthorhalign}
}
\makeatother

\author{\IEEEauthorblockN{1\textsuperscript{st} Chethram Ramoutar}
  \IEEEauthorblockA{\textit{Hunter College} \\
  chethram.ramoutar79@myhunter.cuny.edu}
  \and
  \IEEEauthorblockN{2\textsuperscript{nd} Mark Blinder}
  \IEEEauthorblockA{\textit{Hunter College} \\
  mark.blinder54@myhunter.cuny.edu}
  \linebreakand
  \IEEEauthorblockN{3\textsuperscript{rd} Omar Hegazy}
  \IEEEauthorblockA{\textit{Hunter College} \\
  omar.hegazy00@myhunter.cuny.edu}
  \and
  \IEEEauthorblockN{4\textsuperscript{th} Melanie Vasquez}
  \IEEEauthorblockA{\textit{Hunter College} \\
    melanie.vasquez61@myhunter.cuny.edu}
}

\maketitle

\begin{abstract}
  With the growth of VR devices being more accessible to the masses, it is understood that environment,
  interactivity, and the ability to immerse oneself in a virtual world is a key factor in the success of any
  VR experience. However, achieving this experience is often difficult for many games due to lack of attention to interactivity
  and immersion. In order to combat this, our group seeks to develop a VR game that will use hand input to immerse a user
  in a quest-like game. To achieve immersion and interactivity we will use different tools and methodologies to help the player feel
  like they are apart of the experience.
\end{abstract}

% \begin{IEEEkeywords}
% component, formatting, style, styling, insert
% \end{IEEEkeywords}

\section{Introduction}
The game centers around a lone space traveler who is on a quest to collect different items to build an essential part. As they travel through
different dimensions, the user will experience different terrains, forms of life, and interesting settings. This game will be broken up into four key
areas of travel being: A desert, a forest, a village, and a farm. Each area's focus will allow us to give the user multiple cohesive experiences without
the risk of throwing them off from the main story. This will also allow the developers to test different aspects of each environment to present the idea that
a user truly traveled a different dimension.

Along with the environment, we seek to have non-playable character or NPC interaction that will make each space more lively and natural to the user. However, the main
functionality and interactivity will be with the quests assigned in each area. The user will be able to collect items that relate to their respective quest and explore a majority
of their environment to seek these out. Although it is simple, it can allow the user to immerse themselves without streinuous tasks breaking their immersion. The quests will be entirely focused on
hand input allowing a user to pick up, examine, and collect objects with ease. The user will also be able to look around and navigate the each environment as naturally as possible.

\section{Related Work}
Games that inspired the creation of this project are "The Witness" and "The Forest". "The Witness" uses environment and puzzle interaction in a seemless way. Integration of this in VR has translated well
since the game was first published. By looking at how "The Witness" makes the environment the main aspect, it is something we want to take away from. Despite most if not all of the environment being in low poly 3D graphics, the 
use of ray tracing, saturated colors, and a lack of clutter makes the environment feel alive and natural. The game also uses a first person perspective which is something we want to take away from. The first person perspective allows for a more 
personal and intimate experience with the puzzles that is not often seen in VR games. There is more naturality with this view then with third person perspective.

"The Forest" also uses the previous aspects but in a more realistic and intense setting. The main aspects to take away from is the environmental interaction. Although "The Witness" achieves the appropriate aesthetic while integrating puzzles, "The Forest"
is more realistic with its gameplay. Having the ability to interact with tools such as axes, sticks, and even a bow and arrow brings a lot into the immersion of the game. Movement becomes key in the gameplay which is translated through the well thought out head and hand tracking
that makes every swing or flick feel meaningful in the environment the player is in. Since it is also a survival game, first person perspective brings out the sense of urgency
with every interaction. The game also contains well thought out NPC interactions that are extremely heart racing in a VR environment. By seeing how "The Forest" uses interactivity in its gameplay we want to try and achieve that within our game through our collection
quests. However, we want to keep the calmness and aesthic found in "The Witness" to make the game more enjoyable.

\section{Methodology}
In order to complete the project, it will be done on Unity 2021.3.7f1. All testing will occur with the Oculus Quest 1 and the Oculus Quest 2. Each scene will utilize assets from the Unity Asset Store and any other
source containing appropriate assets. The scenes will be connected through a main hub area which will contain teleportation pads to each area. The user will be able to teleport by standing on a pad and triggering
a script or pressing a button. These portals will also be placed in each area
in order to travel back to allow the user to have a choice on where to go.This will allow us to merge the scenes in a manner that fits with the story but also reduces any streinuous load times that may
occur.

Outside of the hub being worked on, each group member will handle a scene and quest that corresponds with it. The scenes will be split as so:
\begin{itemize}
  \item Desert: Melanie Vasquez
  \item Forest: Mark Blinder
  \item Village: Omar Hegazy
  \item Farm: Chethram Ramoutar
\end{itemize}

Interactivity will be integrated into each scene through the use of the XR rig and XR Interaction Toolkit. This rig will allow for VR testing due to the headset detection built into it and controller support allowing us to add eye and hand detection.
Along with that, the use of the XR Device Simulator will help with non-vr testing to take place throughout the debugging process since we can tweak any features without the full need of a VR headser. The XR Interaction Toolkit will allow for the user to
interact with objects in the scene by using their hands or eyes and will help us with the main portion of the interactivity. Since we are using hands as the main input device, having this allows us to test and enhance interaction experience with either haptic feedback
or visual feedback. This can help build upon each scene once we get into NPC interactions as well and ways the user can have the illusion of communication within each area.

Outside of the interactivity, for immersion we will use assets and different light and rendering techniques to allow a user to be in a space that is not dull. Animations will also be used because it can give the illusion of movement which can add life to each scene.
These animations may be applied to the collected objects, the NPCs, or even the environment itself but overall the main focus is NPC animations. This can be done through Mixamo and scripts that trigger certain movements within a player model.



% \section*{References}

% Please number citations consecutively within brackets \cite{b1}. The
% sentence punctuation follows the bracket \cite{b2}. Refer simply to the reference
% number, as in \cite{b3}---do not use ``Ref. \cite{b3}'' or ``reference \cite{b3}'' except at
% the beginning of a sentence: ``Reference \cite{b3} was the first $\ldots$''

% Number footnotes separately in superscripts. Place the actual footnote at
% the bottom of the column in which it was cited. Do not put footnotes in the
% abstract or reference list. Use letters for table footnotes.

% Unless there are six authors or more give all authors' names; do not use
% ``et al.''. Papers that have not been published, even if they have been
% submitted for publication, should be cited as ``unpublished'' \cite{b4}. Papers
% that have been accepted for publication should be cited as ``in press'' \cite{b5}.
% Capitalize only the first word in a paper title, except for proper nouns and
% element symbols.

% For papers published in translation journals, please give the English
% citation first, followed by the original foreign-language citation \cite{b6}.

% \begin{thebibliography}{00}
%   \bibitem{b1} G. Eason, B. Noble, and I. N. Sneddon, ``On certain integrals of Lipschitz-Hankel type involving products of Bessel functions,'' Phil. Trans. Roy. Soc. London, vol. A247, pp. 529--551, April 1955.
%   \bibitem{b2} J. Clerk Maxwell, A Treatise on Electricity and Magnetism, 3rd ed., vol. 2. Oxford: Clarendon, 1892, pp.68--73.
%   \bibitem{b3} I. S. Jacobs and C. P. Bean, ``Fine particles, thin films and exchange anisotropy,'' in Magnetism, vol. III, G. T. Rado and H. Suhl, Eds. New York: Academic, 1963, pp. 271--350.
%   \bibitem{b4} K. Elissa, ``Title of paper if known,'' unpublished.
%   \bibitem{b5} R. Nicole, ``Title of paper with only first word capitalized,'' J. Name Stand. Abbrev., in press.
%   \bibitem{b6} Y. Yorozu, M. Hirano, K. Oka, and Y. Tagawa, ``Electron spectroscopy studies on magneto-optical media and plastic substrate interface,'' IEEE Transl. J. Magn. Japan, vol. 2, pp. 740--741, August 1987 [Digests 9th Annual Conf. Magnetics Japan, p. 301, 1982].
%   \bibitem{b7} M. Young, The Technical Writer's Handbook. Mill Valley, CA: University Science, 1989.
% \end{thebibliography}
% \vspace{12pt}
% \color{red}
% IEEE conference templates contain guidance text for composing and formatting conference papers. Please ensure that all template text is removed from your conference paper prior to submission to the conference. Failure to remove the template text from your paper may result in your paper not being published.

\end{document}
